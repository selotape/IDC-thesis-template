% XeLaTeX!
% ============================================================ %
% HEBREW support via polyglossia %
% ============================================================ %
\documentclass[12pt]{report} 
\usepackage[utf8x]{inputenc}
\usepackage[english,hebrew]{babel}
\usepackage{hebfont}

\usepackage{polyglossia}
\setmainfont{TeX Gyre Termes}
\newfontfamily\hebrewfont{TeX Gyre Termes}
\defaultfontfeatures{Mapping=tex-text, Scale=MatchLowercase}
\setdefaultlanguage{english}
\setotherlanguage{hebrew}
%\newfontfamily\hebrewfont[Script=Hebrew]{Times New Roman}
% Use \begin{hebrew} block of text \end{hebrew} for paragraphs.
% Use \texthebrew{ } and \textenglish{ } for short texts.
% ============================================================ %

\begin{document}

\section*{\hfill \begin{hebrew}{תקציר}\end{hebrew}}
\begin{hebrew}
משפחות של פונקציות גיבוב נקראות כמעט $k$ בלתי תלויות אם ההתפלגות של הפלטים שלהן על פני כל קבוצה של $k$ קלטים שונים קרובה להתפלגות האחידה בנורמה $L_1$; תכונה זו נקראת גם אי-תלות מוגבלת.
תכונה קרובה לכך היא זו הנקראת כמעט $k$ חוסר-הטיה, המוגדרת ע"י כך שההתפלגות של פלטים של $k$ קלטים שונים קרובה להתפלגות האחידה בנורמה $L_\infty$.

בעבודה זו אנו חוקרים שיטות להגדלה של אי-תלות מוגבלת של משפחות פונקציות גיבוב.
כלומר, בהנתן משפחה של פונקציות גיבוב כמעט $k$ בלתי-תלויה (או בלתי-מוטה), מטרתנו היא ליצור משפחה חדשה של משפחה  שהינה כמעט $k'$ בלתי-תלויה (או בלתי-מוטה) עבור $k' > k$.
ההעתקות (טרנספורמציות) שלנו הינן כלליות במובן שהן מתייחסות אל משפחת פונקציות הגיבוב המקורית כאל קופסא שחורה וברוב המקרים אינן דורשות להניח דבר לגבי משפחה זו מעבר לתכונה הבסיסית של כמעט $k$ חוסר-תלות.
למיטב ידיעתנו, שיטה כזו לא הייתה ידועה קודם לכן.

בכדי להשיג את מטרותינו אנו משתמשים בדגימה חוזרת מהמשפחה המקורית וחיבור של הדגימות באמצעות פונקציה כלשהי.
אנו מזהים שני סוגים של פונקציות שיעילות למטרה זו: הסוג הראשון הוא פונקציות אשר מקטימנות את המרחק בנורמה $L_\infty$ בין ההתפלגות של הפלטים להתפלגות האחידה; הסוג השני מאפשר לנו להגדיל את פרמטר האי-תלות $k$.
לבסוף, אנו מחברים את שני סוגי הפונקציות הללו בתהליך איטרטיבי אשר בסופו של דבר משיג את התכונות הדרושות.
\end{hebrew}
\newpage

\begin{hebrew}
עבודה זו בוצעה בהדרכתו של אלון רוזן.
העבודה מבוססת על מחקר משותף עם אנדרי בוגדנוב ואלון רוזן.
\end{hebrew}

\newpage

\begin{titlepage}
	\centering
	\includegraphics[width=0.4\textwidth]{IDC_logo_hebrew}\par\vspace{2cm}
	{\huge \begin{hebrew}{המרכז הבינתחומי הרצליה}\end{hebrew} \par}
	{\Large \begin{hebrew}{בית הספר ארזי למדעי המחשב}\end{hebrew} \par}
	{\Large \begin{hebrew}{התכנית לתואר שני (M.Sc.) - מסלול מחקרי}\end{hebrew} \par}
	
	\vspace{1cm}
	
	\vspace{1.5cm}
	{\Huge \begin{hebrew} שיטה כללית להרחבה של אי-תלות מוגבלת \end{hebrew} \par}
	\vspace{3cm}
	{\large \begin{hebrew}{מאת}\end{hebrew}\par}
	{\large\bfseries \begin{hebrew} ארבל דויטש פלד \end{hebrew} \par}
	
	\vspace{2cm}
	
	{\begin{hebrew}{עבודת תזה המוגשת כחלק מהדרישות לשם קבלת תואר מוסמך M.Sc.}\end{hebrew}\par}
	{\begin{hebrew}{במסלול המחקרי בבית הספר אפי ארזי למדעי המחשב, המרכז הבינתחומי הרצליה}\end{hebrew}\par}
	
	\vfill
	
	% Bottom of the page
	{\large\begin{hebrew}{ספטמבר 2016}\end{hebrew}\par}
\end{titlepage}

\end{document}